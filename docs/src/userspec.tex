\section{Felhasználói dokumentáció}

\subsection{A program célja}
A felhasználó polinom függvényeket ad meg a programnak, mely annak grafikonját ascii art formájában
vagy egy fájlba menti, vagy interaktív módban megjeleníti a konzolon.

\subsection{Megkötések}
A program csak polinom függvényeket tud értelmezni, szögfüggvények / hatványfüggvényekre hibát ír ki.


\subsection{A program által elfogadott kapcsolók}
A programot az alábbi sematika szerint kell futtatni:
\begin{verbatim}
    plottr [kapcsolók] <függvény>
\end{verbatim}
A kapcsolókat tetszőleges sorrendben, opcionálisan megadhatók.

\begin{table}[h]
\centering
\resizebox{\textwidth}{!}{%
\begin{tabular}{|c|c|c|c|}
Név & Flag & Rövidítés & Magyarázat \\
\hline
Interaktív mód & --tui & -t & Interaktív módba állítja a programot \\
Mozgatás & --move <x:num> <y:num> & -m & Adott x, y koordinátákra mozgatja a nézetet \\
Elmozdulás & --offset <x:num> <y:num> & -o & A jelenlegi nézettől mérten relatívan mozog \\
Méretarány & --scale <w:num> <h:num> & -s & A kijelző szélességét és magasságát adjuk meg \\
Információ kijelző & --info & -i & A függvényről jelenít meg információkat \\
Függvény megadása & --plot <exp...:number> & -p & A függvény eggyütthatóit lehet beállítani \\
Stílusok & --style <basic|ascii|unicode> & -s & A függvény megjelenítését állíthatjuk \\
Mentés & --out <f:file> & -o & Kimenti a függvényt fájlba \\
Segítség & --help & -h & A program használatáról ír ki információkat \\
\end{tabular}
}%
\caption{A program által elfogadott kapcsolók}
\label{tab:student_scores}
\end{table}

