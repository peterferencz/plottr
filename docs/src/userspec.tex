\section*{Felhasználói dokumentáció}

\subsection*{A program célja}
A felhasználó polinom függvényeket ad meg a programnak, mely annak grafikonját ascii art formájában
vagy egy fájlba menti, vagy interaktív módban megjeleníti a konzolon.

\subsection*{Megkötések}
A program csak polinom függvényeket tud értelmezni, szögfüggvények / hatványfüggvényekre hibát ír ki.


\subsection*{A program által elfogadott kapcsolók}
A programot az alábbi sematika szerint kell futtatni:
\begin{verbatim}
    plottr [kapcsolók] <függvény>
\end{verbatim}
A kapcsolókat tetszőleges sorrendben, opcionálisan megadhatók:\\
 - interactive: a felhasználó egy interaktív tui felületen kezelheti a programot\\
 - offset <x:num> <y:num>: Az origótól való x és y eltolás\\
 - o / out <fájl:file>: A függvény képének (txt formátumban ) mentési helye.
        Ha nincs megadva, akkor a standard kimenetre ír\\
 - s / scale <width:num>: A vízszintes tengely szélessége\\
 - h / help: A program futtatását kiíró kapcsoló\\

\subsection*{Interaktív mód parancsai}
Ha a programot az -interactive kapcsolóval indítjuk el, akkor futás közben
további parancsokat adhatunk a programnak, melyek az alábbiak\\
 - move <x:num> <y:num>: Adott x és y koordinátákra mozog\\
 - scale <w:num>: A vízszintes tengely hosszát állítja be\\
 - dx <x:num>: az x tengelyen mozog\\
 - dy <y:num>: az y tengelyen mozog\\
 - save / out <fájl:file>: A függvény képét (txt formátumban) elmenti\\
 - plot <polinom:expr>: Új grafikont ábrázol\\
 - plot: A grafikon nézetre vált\\
 - info: A grafikonról jelenít meg információkat\\